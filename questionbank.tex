% !TeX root = questionnaire.tex

%%% snippets: mcsa[h|mc], gra, igs

% chktex-file 1, chktex-file 8, chktex-file 36

\tikzset{
	parallelograms1/.pic={
			\begin{scope}[radius=2pt]
				\draw (0,0) coordinate (A) -- ++(1,3) coordinate (B) -- ++(4,0.5) coordinate (C) -- ++(-1,-3) coordinate (D) -- cycle;
				\node[below left] at (A) {A};
				\node[above left] at (B) {B};
				\node[below right] at (C) {C};
				\node[below right] at (D) {D};
				\draw (A) ++(7,2) coordinate (E) -- ++(1,3) coordinate (F) -- ++(4,0.5) coordinate (G) -- ++(-1,-3) coordinate (H) -- cycle;
				\node[above left] at (E) {E};
				\node[above left] at (F) {F};
				\node[above right] at (G) {G};
				\node[below right] at (H) {H};
				\foreach \point in {A,B,C,D,E,F,G,H}
				\fill[black] (\point) circle;
				\draw (B) -- (F);
				\draw (C) -- (G);
				\draw (A) -- (E);
				\draw (D) -- (H);
				\draw[dashed, name path=AC] (A) -- (C);
				\draw[dashed, name path=BD] (B) -- (D);
				\draw[dashed, name path=EG] (E) -- (G);
				\draw[dashed, name path=FH] (F) -- (H);
				\fill[black, name intersections={of=AC and BD}] (intersection-1) circle node[above] {I};
				\fill[black, name intersections={of=EG and FH}] (intersection-1) circle node[right] {J};
			\end{scope}
		},
	parallelograms2/.pic={
			\begin{scope}[radius=2pt]
				\draw (0,0) coordinate (A) -- ++(-1,3) coordinate (B) -- ++(4,0.5) coordinate (C) -- ++(1,-3) coordinate (D) -- cycle;
				\node[below left] at (A) {A};
				\node[above left] at (B) {B};
				\node[above] at (C) {C};
				\node[below right] at (D) {D};
				\draw (A) ++(7,2) coordinate (E) -- ++(-1,3) coordinate (F) -- ++(4,0.5) coordinate (G) -- ++(1,-3) coordinate (H) -- cycle;
				\node[below] at (E) {E};
				\node[above left] at (F) {F};
				\node[above right] at (G) {G};
				\node[right] at (H) {H};
				\foreach \point in {A,B,C,D,E,F,G,H}
				\fill[black] (\point) circle;
				\draw (B) -- (F);
				\draw (C) -- (G);
				\draw (A) -- (E);
				\draw (D) -- (H);
				\draw[dashed, name path=AC] (A) -- (C);
				\draw[dashed, name path=BD] (B) -- (D);
				\draw[dashed, name path=EG] (E) -- (G);
				\draw[dashed, name path=FH] (F) -- (H);
				\fill[black, name intersections={of=AC and BD}] (intersection-1) circle node[above] {I};
				\fill[black, name intersections={of=EG and FH}] (intersection-1) circle node[below] {J};
			\end{scope}
		}
}

\element{q01}{
	\begin{question}{q01.1}
		Sur cette figure formée de parallélogrammes juxtaposés, quelle proposition est un représentant de \(\vec{AI} + \vec{FJ} - \vec{GH}\)~?
		\begin{center}
			\begin{tikzpicture}[radius=2pt]
				\pic at (0,0) {parallelograms1};
			\end{tikzpicture}
		\end{center}
		\begin{multicols}{2}\AMCBoxedAnswers{}
			\begin{choices}
				\wrongchoice{\(\vec{AG}\)}
				\wrongchoice{\(\vec{AH}\)}
				\wrongchoice{\(\vec{BC}\)}
				\correctchoice{\(\vec{EG}\)}
				\lastchoices{}\columnbreak{}
				\wrongchoice{aucune}
				\wrongchoice{toutes}
				\wrongchoice{manque}
				\wrongchoice{absurdité}
			\end{choices}
		\end{multicols}
	\end{question}
}

\element{q01}{
	\begin{question}{q01.2}
		Sur cette figure formée de parallélogrammes juxtaposés, quelle proposition est un représentant de \(\vec{BI} + \vec{JH} - \vec{CD}\)~?
		\begin{center}
			\begin{tikzpicture}
				\pic at (0,0) {parallelograms1};
			\end{tikzpicture}
		\end{center}
		\begin{multicols}{2}\AMCBoxedAnswers{}
			\begin{choices}
				\wrongchoice{\(\vec{BD}\)}
				\wrongchoice{\(\vec{BG}\)}
				\wrongchoice{\(\vec{BH}\)}
				\correctchoice{\(\vec{FG}\)}
				\lastchoices{}\columnbreak{}
				\wrongchoice{aucune}
				\wrongchoice{toutes}
				\wrongchoice{manque}
				\wrongchoice{absurdité}
			\end{choices}
		\end{multicols}
	\end{question}
}

\element{q02}{
	\begin{question}{q02.1}
		Sur cette figure formée de parallélogrammes juxtaposés, quelle proposition est un représentant de \(\vec{AC} + \vec{CD}\)~?
		\begin{center}
			\begin{tikzpicture}[radius=2pt]
				\pic at (0,0) {parallelograms2};
			\end{tikzpicture}
		\end{center}
		\begin{multicols}{2}\AMCBoxedAnswers{}
			\begin{choices}
				\wrongchoice{\(\vec{BG}\)}
				\wrongchoice{\(\vec{CF}\)}
				\wrongchoice{\(\vec{EG}\)}
				\wrongchoice{\(\vec{FH}\)}
				\lastchoices{}\columnbreak{}
				\correctchoice{aucune}
				\wrongchoice{toutes}
				\wrongchoice{manque}
				\wrongchoice{absurdité}
			\end{choices}
		\end{multicols}
	\end{question}
}

\element{q02}{
	\begin{question}{q02.2}
		Sur cette figure formée de parallélogrammes juxtaposés, quelle proposition est un représentant de \(\vec{AI} + \vec{IC}\)~?
		\begin{center}
			\begin{tikzpicture}
				\pic at (0,0) {parallelograms2};
			\end{tikzpicture}
		\end{center}
		\begin{multicols}{2}\AMCBoxedAnswers{}
			\begin{choices}
				\wrongchoice{\(\vec{AG}\)}
				\wrongchoice{\(\vec{AH}\)}
				\wrongchoice{\(\vec{BD}\)}
				\wrongchoice{\(\vec{FH}\)}
				\lastchoices{}\columnbreak{}
				\correctchoice{aucune}
				\wrongchoice{toutes}
				\wrongchoice{manque}
				\wrongchoice{absurdité}
			\end{choices}
		\end{multicols}
	\end{question}
}

\element{q03}{
	\begin{question}{q03.1}
		Quelles sont les composantes \(x\) et \(y\) du vecteur \(\vec{u}\) si sa norme vaut \qty{\directlua{tex.print(u_norm)}}{\meter}~?
		\directlua{
			u_norm = 2
			u_angle = 40
			u_angle_rad = math.rad(u_angle)
			u_x = round(u_norm * math.cos(u_angle_rad), 2)
			u_y = round(u_norm * math.sin(u_angle_rad), 2)
			f1_x = u_y
			f1_y = u_x
			f2_x = round(math.cos(u_angle_rad), 2)
			f2_y = round(math.sin(u_angle_rad), 2)
			f3_x = u_norm
			f3_y = round(u_norm * math.sin(u_angle_rad), 2)
		}
		\begin{center}
			\begin{tikzpicture}
				\draw[->] (0,0) -- (\directlua{tex.print(u_x + 4)},0) node[right] {\(x\)};
				\draw[->] (0,0) -- (0,\directlua{tex.print(u_y + 2)}) node[above] {\(y\)};
				\draw[very thick,-Stealth] (2,1) coordinate(B) -- +(\directlua{tex.print(u_angle)}:\directlua{tex.print(u_norm)}) coordinate(A);
				\node[above right] at (A) {\(\vec{u}\)};
				\draw[help lines,dashed] (B) -- ++(0,\directlua{tex.print(u_y)}) coordinate(C);
				\pic["\ang{\directlua{tex.print(90 - u_angle)}}",draw,angle eccentricity=2] {angle};
			\end{tikzpicture}
		\end{center}
		\begin{multicols}{3}\AMCBoxedAnswers{}
			\begin{choices}
				\correctchoice{\(\begin{pmatrix}\qty{\directlua{tex.print(u_x)}}{\meter} \\ \qty{\directlua{tex.print(u_y)}}{\meter}\end{pmatrix}\)}
				\wrongchoice{\(\begin{pmatrix}\qty{\directlua{tex.print(f1_x)}}{\meter} \\ \qty{\directlua{tex.print(f1_y)}}{\meter}\end{pmatrix}\)}
				\wrongchoice{\(\begin{pmatrix}\qty{\directlua{tex.print(f2_x)}}{\meter} \\ \qty{\directlua{tex.print(f2_y)}}{\meter}\end{pmatrix}\)}
				\wrongchoice{\(\begin{pmatrix}\qty{\directlua{tex.print(f3_x)}}{\meter} \\ \qty{\directlua{tex.print(f3_y)}}{\meter}\end{pmatrix}\)}
				\lastchoices{}\columnbreak{}
				\wrongchoice{aucune}
				\wrongchoice{toutes}
				\wrongchoice{manque}
				\wrongchoice{absurdité}
			\end{choices}
		\end{multicols}
	\end{question}
}

\element{q03}{
	\begin{question}{q03.2}
		Quelles sont les composantes \(x\) et \(y\) du vecteur \(\vec{u}\) si sa norme vaut \qty{2}{\meter}~?
		\begin{center}
			\begin{tikzpicture}
				\draw[->] (0,0) -- (7,0) node[right] {\(x\)};
				\draw[->] (0,0) -- (0,4) node[above] {\(y\)};
				\draw[very thick,-Stealth] (2,1) coordinate(B) -- +(30:3) coordinate(A);
				\node[above right] at (A) {\(\vec{u}\)};
				\draw[help lines,dashed] (B) -- ++(0,2) coordinate(C);
				\pic["\ang{60}",draw,angle eccentricity=2] {angle};
			\end{tikzpicture}
		\end{center}
		\begin{multicols}{3}\AMCBoxedAnswers{}
			\begin{choices}
				\wrongchoice{\(\begin{pmatrix}\qty{0.87}{\meter} \\ \qty{0.5}{\meter}\end{pmatrix}\)}
				\wrongchoice{\(\begin{pmatrix}\qty{1}{\meter} \\ \qty{1.73}{\meter}\end{pmatrix}\)}
				\correctchoice{\(\begin{pmatrix}\qty{1.73}{\meter} \\ \qty{1}{\meter}\end{pmatrix}\)}
				\wrongchoice{\(\begin{pmatrix}\qty{2}{\meter} \\ \qty{1}{\meter}\end{pmatrix}\)}
				\lastchoices{}\columnbreak{}
				\wrongchoice{aucune}
				\wrongchoice{toutes}
				\wrongchoice{manque}
				\wrongchoice{absurdité}
			\end{choices}
		\end{multicols}
	\end{question}
}

\element{q04}{
	\begin{question}{q04.1}
		Soit les vecteurs \(\vec{A}\) et \(\vec{B}\) situés dans le plan \(xy\).
		La norme de \(\vec{A}\) vaut \qty{3.2}{\meter} et l'angle \(\theta_A = \ang{45}\).
		La norme de \(\vec{B}\) vaut \qty{2.4}{\meter} et l'angle \(\theta_B = \ang{290}\).
		Que vaut \(A \cdot B\)~?
		\begin{multicols}{2}\AMCBoxedAnswers{}
			\begin{choices}
				\wrongchoice{\qty{-6.96}{\meter}}
				\correctchoice{\qty{-3.25}{\meter}}
				\wrongchoice{\qty{2.62}{\meter}}
				\wrongchoice{\qty{5.43}{\meter}}
				\lastchoices{}\columnbreak{}
				\wrongchoice{aucune}
				\wrongchoice{toutes}
				\wrongchoice{manque}
				\wrongchoice{absurdité}
			\end{choices}
		\end{multicols}
	\end{question}
}

\element{q04}{
	\begin{question}{q04.2}
		Soit les vecteurs \(\vec{A}\) et \(\vec{B}\) situés dans le plan \(xy\).
		La norme de \(\vec{A}\) vaut \qty{3.2}{\meter} et l'angle \(\theta_A = \ang{60}\).
		La norme de \(\vec{B}\) vaut \qty{2.4}{\meter} et l'angle \(\theta_B = \ang{290}\).
		Que vaut \(A \cdot B\)~?
		\begin{multicols}{2}\AMCBoxedAnswers{}
			\begin{choices}
				\correctchoice{\qty{-30.25}{\meter}}
				\wrongchoice{\qty{-10.51}{\meter}}
				\wrongchoice{\qty{19.75}{\meter}}
				\wrongchoice{\qty{26.60}{\meter}}
				\lastchoices{}\columnbreak{}
				\wrongchoice{aucune}
				\wrongchoice{toutes}
				\wrongchoice{manque}
				\wrongchoice{absurdité}
			\end{choices}
		\end{multicols}
	\end{question}
}
