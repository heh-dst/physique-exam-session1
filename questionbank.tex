% !TeX root = questionnaire.tex

%%% snippets: mcsa[h|mc], gra, igs

% chktex-file 1, chktex-file 8, chktex-file 36

\tikzset{
	parallelograms1/.pic={
			\begin{scope}[radius=2pt]
				\draw (0,0) coordinate (A) -- ++(1,3) coordinate (B) -- ++(4,0.5) coordinate (C) -- ++(-1,-3) coordinate (D) -- cycle;
				\node[below left] at (A) {A};
				\node[above left] at (B) {B};
				\node[below right] at (C) {C};
				\node[below right] at (D) {D};
				\draw (A) ++(7,2) coordinate (E) -- ++(1,3) coordinate (F) -- ++(4,0.5) coordinate (G) -- ++(-1,-3) coordinate (H) -- cycle;
				\node[above left] at (E) {E};
				\node[above left] at (F) {F};
				\node[above right] at (G) {G};
				\node[below right] at (H) {H};
				\foreach \point in {A,B,C,D,E,F,G,H}
				\fill[black] (\point) circle;
				\draw (B) -- (F);
				\draw (C) -- (G);
				\draw (A) -- (E);
				\draw (D) -- (H);
				\draw[dashed, name path=AC] (A) -- (C);
				\draw[dashed, name path=BD] (B) -- (D);
				\draw[dashed, name path=EG] (E) -- (G);
				\draw[dashed, name path=FH] (F) -- (H);
				\fill[black, name intersections={of=AC and BD}] (intersection-1) circle node[above] {I};
				\fill[black, name intersections={of=EG and FH}] (intersection-1) circle node[right] {J};
			\end{scope}
		},
	parallelograms2/.pic={
			\begin{scope}[radius=2pt]
				\draw (0,0) coordinate (A) -- ++(-1,3) coordinate (B) -- ++(4,0.5) coordinate (C) -- ++(1,-3) coordinate (D) -- cycle;
				\node[below left] at (A) {A};
				\node[above left] at (B) {B};
				\node[above] at (C) {C};
				\node[below right] at (D) {D};
				\draw (A) ++(7,2) coordinate (E) -- ++(-1,3) coordinate (F) -- ++(4,0.5) coordinate (G) -- ++(1,-3) coordinate (H) -- cycle;
				\node[below] at (E) {E};
				\node[above left] at (F) {F};
				\node[above right] at (G) {G};
				\node[right] at (H) {H};
				\foreach \point in {A,B,C,D,E,F,G,H}
				\fill[black] (\point) circle;
				\draw (B) -- (F);
				\draw (C) -- (G);
				\draw (A) -- (E);
				\draw (D) -- (H);
				\draw[dashed, name path=AC] (A) -- (C);
				\draw[dashed, name path=BD] (B) -- (D);
				\draw[dashed, name path=EG] (E) -- (G);
				\draw[dashed, name path=FH] (F) -- (H);
				\fill[black, name intersections={of=AC and BD}] (intersection-1) circle node[above] {I};
				\fill[black, name intersections={of=EG and FH}] (intersection-1) circle node[below] {J};
			\end{scope}
		}
}

\element{q01}{
	\begin{question}{q01.1}
		Sur cette figure formée de parallélogrammes juxtaposés, quelle proposition est un représentant de \(\vec{AI} + \vec{FJ} - \vec{GH}\)~?
		\begin{center}
			\begin{tikzpicture}[radius=2pt]
				\pic at (0,0) {parallelograms1};
			\end{tikzpicture}
		\end{center}
		\begin{multicols}{2}\AMCBoxedAnswers{}
			\begin{choices}
				\wrongchoice{\(\vec{AG}\)}
				\wrongchoice{\(\vec{AH}\)}
				\wrongchoice{\(\vec{BC}\)}
				\correctchoice{\(\vec{EG}\)}
				\lastchoices{}\columnbreak{}
				\wrongchoice{aucune}
				\wrongchoice{toutes}
				\wrongchoice{manque}
				\wrongchoice{absurdité}
			\end{choices}
		\end{multicols}
	\end{question}
}

\element{q01}{
	\begin{question}{q01.2}
		Sur cette figure formée de parallélogrammes juxtaposés, quelle proposition est un représentant de \(\vec{BI} + \vec{JH} - \vec{CD}\)~?
		\begin{center}
			\begin{tikzpicture}
				\pic at (0,0) {parallelograms1};
			\end{tikzpicture}
		\end{center}
		\begin{multicols}{2}\AMCBoxedAnswers{}
			\begin{choices}
				\wrongchoice{\(\vec{BD}\)}
				\wrongchoice{\(\vec{BG}\)}
				\wrongchoice{\(\vec{BH}\)}
				\correctchoice{\(\vec{FG}\)}
				\lastchoices{}\columnbreak{}
				\wrongchoice{aucune}
				\wrongchoice{toutes}
				\wrongchoice{manque}
				\wrongchoice{absurdité}
			\end{choices}
		\end{multicols}
	\end{question}
}

\element{q02}{
	\begin{question}{q02.1}
		Sur cette figure formée de parallélogrammes juxtaposés, quelle proposition est un représentant de \(\vec{AC} + \vec{CD}\)~?
		\begin{center}
			\begin{tikzpicture}[radius=2pt]
				\pic at (0,0) {parallelograms2};
			\end{tikzpicture}
		\end{center}
		\begin{multicols}{2}\AMCBoxedAnswers{}
			\begin{choices}
				\wrongchoice{\(\vec{BG}\)}
				\wrongchoice{\(\vec{CF}\)}
				\wrongchoice{\(\vec{EG}\)}
				\wrongchoice{\(\vec{FH}\)}
				\lastchoices{}\columnbreak{}
				\correctchoice{aucune}
				\wrongchoice{toutes}
				\wrongchoice{manque}
				\wrongchoice{absurdité}
			\end{choices}
		\end{multicols}
	\end{question}
}

\element{q02}{
	\begin{question}{q02.2}
		Sur cette figure formée de parallélogrammes juxtaposés, quelle proposition est un représentant de \(\vec{AI} + \vec{IC}\)~?
		\begin{center}
			\begin{tikzpicture}
				\pic at (0,0) {parallelograms2};
			\end{tikzpicture}
		\end{center}
		\begin{multicols}{2}\AMCBoxedAnswers{}
			\begin{choices}
				\wrongchoice{\(\vec{AG}\)}
				\wrongchoice{\(\vec{AH}\)}
				\wrongchoice{\(\vec{BD}\)}
				\wrongchoice{\(\vec{FH}\)}
				\lastchoices{}\columnbreak{}
				\correctchoice{aucune}
				\wrongchoice{toutes}
				\wrongchoice{manque}
				\wrongchoice{absurdité}
			\end{choices}
		\end{multicols}
	\end{question}
}
